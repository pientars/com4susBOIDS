\section{Motivation}
Human development can infringe upon important bird habitats, reducing biodiversity. In particular, there is evidence that migratory birds are less likely to inhabit areas of increasing human development~\cite{riparian}. Thus, to minimize disruption of migration, we should be aware of prominent migration corridors so that we can prioritize them in conservation efforts. The 'sources' and 'sinks' of this network are also key candidates for conservation efforts. Migration pattern awareness also allows us more insight into the potential spread of bird-borne viruses.

As a crowd-sourced research project, eBird presents a more horizontal data collection effort: instead of creating a model built from a small number of researchers' observations, we can aggregate data from thousands of observers around the United States. While this creates unique challenges due to spatial bias and inexperienced observers, it also allows for much larger statistical data sets to draw from. 

Creating a migration model based on eBird data theoretically allows us to study \textit{real-time}, \textit{crowd-sourced} temporal migration patterns, rather than static models aggregated over many years. This gives us the additional potential to see changes in migration patterns on a smaller, more agile time scale.