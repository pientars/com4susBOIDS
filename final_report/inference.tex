\section{Network Inference} 
The statistical approach to graph inference is a well researched area with existing effective techniques \cite{AlbertMechanics}. Because the nature of this research problem involves inferring a network from a time-varying signal, we will leverage the work of Kramer et al. who proposed a state of the art method for statistically inferring an underlying functional network from a time-varied signal~\cite{kramer}. In this case, we attempt to establish the existence of signal propagation from one node to another, with a certain degree of confidence. 

Kramer, et. al present three methods of increasing sophistication (and increasing computational intensity) for inferring a network from a set of nodes and corresponding time-series data. The final, most sophisticated method is computationally infeasible for large data sets (requiring a massive number of FFTs compounded by a massive number of bootstrap runs), so we explore the effectiveness of the initial two methods in our work. 

Network inference generally takes three steps~\cite{kramer}:
\begin{enumerate}
\item Define a coupling measure between pairs of nodes (i.e, cross-correlation)
\item Utilize a significance test to detect statistically significant couplings
\item Integrate with further significance tests, if necessary
\end{enumerate}

In the approach taken by Kramer, et. al, the coupling measure is cross correlation:

\[ C_{ij}[\tau] = \frac{1}{\hat\sigma_i\hat\sigma_j (n-2\tau)}\sum_{t=1}^{n-\tau}(x_i[t] - \bar x_i )(x_j[t+\tau] - \bar x_j) \]

The extrema from this correlation will likely be produced in such a way that normal-based significance tests will fail. Thus, we compute new values using the Fisher transformation, which we then perform a significance test on:

\[ C_{ij}^F [\tau] = \frac{1}{2} \ln \frac{1+C_{ij}[\tau]}{1-C_{ij}[\tau]} \]

\[ P[z] = Pr\{z_{ij}^F \leq z\} \approx \exp\{-2 \exp [-a_n(z-b_n)]\} \]
\[ z_{ij}^F=\frac{s_{ij}^F}{var(C_{ij}^F)^\frac{1}{2}} \]
\[ a_n = \sqrt{2 \ln n} \]
\[ b_n = a_n - (2a_n)^{-1}(\ln \ln n + \ln 4\ pi) \]

A final detail is that we will need to look at `slices' of the time-series data and perform network inference for a set number of time periods (to account for different migratory behaviors in different seasons). The final network can be unweighted (where an edge exists if its weight is above a certain threshold), or weighted by the confidence that it is in the migratory network, to identify important corridors, sources, and sinks using a flow analysis. 

As mentioned above, we avoid looking at all possible $n \choose 2$ edges by only looking at the 8-neighborhood of each node, for a total number of correlation computations of $8n$.
